\documentclass[12pt, letterpaper,titlepage]{article}
\usepackage[utf8]{inputenc}
\usepackage{graphicx}
\usepackage{hyperref}
\usepackage{amsmath}
\usepackage{amssymb, bm, amsthm}

\def\bf{\mathbf}
\newcommand{\vect}[1]{\boldsymbol{\mathbf{#1}}}
\newcommand{\mat}[1]{\begin{bmatrix} #1 \end{bmatrix}}

\title{ME 601 Project Report \\
\textbf{Rendezvous \& Proximity Operations}}
\author{Ian Ruh (\href{mailto:iruh@wisc.edu}{iruh@wisc.edu})}
\date{May 1, 2022}

\begin{document}
\maketitle

% - Introduction
% - Background
%   - Applications
%   - Coordinates
%   - Dynamics
% - Controllers
%   - Stabilizing Infinite LQR
%   - Trajectory Tracking Infinite LQR
%   - Trajectory Tracking non-linear infinite LQR
% - Experiments
%   - Simulation setup
%   - Local stabilization
%   - Box experiments'
% - Discussion
% - References

\section{Overview}

I'll be utilizing the material covered in class to analyze the control
and relative dynamics of two spacecraft in orbit around earth. This is an
area of interest due to both its current usage in vehicle docking to space
stations, as well as its future applications to in-orbit debris collection and
in-orbit assembly.

In order to simplify the analysis, several assumptions on the orbital
parameters of the two spacecrafts, the configuration of actuators, the ideal
behavior of the actuators, the control of the spacecraft attitude and knowledge
of the spacecraft state will be made. These assumptions and their impact are
included in the background section.

\section{Background}

The rendevous of two space craft, or more generally the relative positioning of
two spacecraft in orbit, is a complex problem due to the dynamics involved. The
first step in approaching the problem is to select a coordinate system in which
the relative dynamics will be studied.

In the rest of this document, we
identify two space crafts: the target and the chaser. The target is assumed to
have attitude control such that it maintains a consistent orientation with
respect to earth's horizon, but is otherwise inert. The chaser is the vehicle
that we will focus on controlling the position of to rendezvous
with the target.

We identify the Radial-Transverse-Normal (RTN) coordinate frame centered with
the origin at the center of mass of the target vehicle. The three axes are
aligned in the radial direction, along the vector from the origin of the center
of the earth; the transverse direction, parallel to the velocity vector, and
thus along the orbital path; and the normal direction, parralel to the orbit's
angular momentum vector.

Within this coordinate frame the following relative dynamics are derived in
\cite{sullivan_comprehensive_2017}:

\begin{equation}
    \label{full_dynamics}
    \begin{split}
        \ddot{x} - 2\dot{f}_c\dot{y} - \ddot{f}_c y - \dot{f}_c^2 x & =
            -\frac{\mu(r_c + x)}{\left[ (r_c+x)^2 + y^2 + z^2\right]^{3/2}} +
            \frac{\mu}{r_c^2} + d_R \\
        \ddot{y} + 2\dot{f}_c\dot{x} + \ddot{f}_c x - \dot{f}_c^2 y & =
            -\frac{\mu y}{\left[ (r_c+x)^2 + y^2 + z^2\right]^{3/2}} + d_T \\
        \ddot{z} & = -\frac{\mu z}{\left[ (r_c+x)^2 + y^2 + z^2\right]^{3/2}} +
            d_N
    \end{split}
\end{equation}

Where $f_c$ is the target's true anomaly, $r_c$ is the target's radius, and
$\vec{d} = [d_R, d_T, d_N]^T$ represents the relative disturbing accelerations
between the target and the chaser in the target's RTN frame.

The first simplifying assumption we make is on the eccentricity of the target's
orbit. If the target's orbit is near circular (the eccentricity is near 0),
then $\dot{f}_c = \text{mean motion} = \sqrt{\frac{\mu}{a^3}}$, where $\mu$ is
the earth's gravitational parameter ($\approx 3.986\cdot10^{14} \text{
m}^2/\text{s}$), and $a$ is the orbit's semi-major axis (SMA). Thus,
$\dot{f}_c$ is constant and $\ddot{f}_c = 0$. This leads to the following
simplified dynamics:

\begin{equation}
    \label{circular_dynamics}
    \begin{split}
        \ddot{x} & =
            -\frac{\mu(r_c + x)}{\left[ (r_c+x)^2 + y^2 + z^2\right]^{3/2}} +
            \frac{\mu}{r_c^2} + 2n\dot{y} + n^2 x + d_R \\
        \ddot{y} & = -\frac{\mu y}{\left[ (r_c+x)^2 + y^2 + z^2\right]^{3/2}} +
            2n\dot{x} + n^2 y + d_T \\
        \ddot{z} & = -\frac{\mu z}{\left[ (r_c+x)^2 + y^2 + z^2\right]^{3/2}} +
            d_N
    \end{split}
\end{equation}

Linearizing this around the target's position, we arrive at the
following linear dynamics \cite{cli_clohessy_nodate}:

\begin{equation}
    \label{linear_dynamics}
    \begin{split}
        \ddot{x} & = 3n^2 x + 2n\dot{y}  + d_R \\
        \ddot{y} & = -2n\dot{x} + d_T \\
        \ddot{z} & = -n^2z + d_N
    \end{split}
\end{equation}

\subsection{Other Assumptions}

The reality of controlling a spacecraft involves complex actuators including
electric propulsion, mono-propellant attitude control systems, reaction wheels,
control moment gyros, magnetic torquers, and others. For the purposes of this
project, only the dynamics due to gravity and idealized forces in the RTN frame
are considered to simplify the analysis. Therefore, we are neglecting the
attitude control of the chaser and the arrangement (and therefore selection of
thrusters) such that we have have a force in the RTN frame as our only control
input.

\section{Controllers}

\subsection{Infinte LQR Linearized Stabilizing Controller}

I implemented an infinte LQR controller that 

\subsection{Infinte LQR Linearized Trajectory Tracking Controller}

We deriving the expression for our error dynamics using the linearized system
dynamics. We define $e(t) = x(t) - x_d(t)$, where $x_d$ is our desired
trajectory, and $v(t) = u(t) - u_d(t)$, where $u_d$ is our desired control.

\begin{equation}
    \label{linearized_error_dynamics}
    \begin{split}
        \dot{e} & = \dot{x} - \dot{x}_d \\
                & = f(x, u) - f(x_d, u_d) \\
                & = f(x_d + e, u_d + v) - f(x_d, u_d) \\
                & = (\bf{A}(x_d + e) + \bf{B}(u_d + v)) - (\bf{A}x_d +
                    \bf{B}u_d) \\
                & = \bf{A}e + \bf{B}v
    \end{split}
\end{equation}

For simplicity, we let $v(t) = \vec{0}.$

\bibliography{mybibfile}
\bibliographystyle{plain}

\end{document}

